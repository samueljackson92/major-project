\chapter{Third-Party Code and Libraries}

\begin{table}[H]
\centering
\begin{tabular}{l p{14cm}}
\toprule
         name & description \\
\midrule
        click & Command line interface library used to create the various CLI tools supplied with the package created for this project.\\ \hline
     coverage & Python library used to measure the coverage provided by the unit tests. This was used both by the developer and automatically run by the build server \\ \hline
   matplotlib & Matplotlib is an general purpose 2D and 3D plotting library. This library has been heavily used both as a component of the pandas library, and in its own right to generate most of the plots shown in this document. \\ \hline
        medpy & Library which provides functions for reading DICOM format images \\ \hline
         nose & Python unit testing tool. This provides a suite of test helpers and assertion functions as well as a command line program used to run the project's unit \& regression tests \\ \hline
        numpy & General purpose, fast and efficient array manipulation library. This is a core dependancy of scipy, pandas, and the scikit libraries. \\ \hline
       pandas & Pandas provides high level data analysis and manipulation tools. This project is heavily dependant on pandas for its database-esque operations, plotting routines, and I/O routines.\\ \hline
 scikit-image & scikit-image is built on top of the scipy library and provides many of the higher level image analysis functions such as loading and transforming images, as well as the implementation of the GLCM matrices used in the project.\\ \hline
 scikit-learn & The scikit-learn library provided implementations of the manifold learning algorithms used in the project. This includes the implementations of t-SNE, LLE and Isomap. Other uses of the library were for pairwise distance computations and $k$-means clustering. \\ \hline
        scipy & Scipy is a collection of mathematical, scientific, and engieering packages. Scipy is a core dependancy of many of the other third-party libraries used in this project. It has also been heavily used within the project itself.\\ \hline
      seaborn & Seaborn is another plotting library built on top of matplotlib. It provides some additional plotting functionality not present in matplotlib itself.\\ \hline
       sphinx & Sphinx is a library which can be used to automatically generate documentation from the docstrings of Python code.\\
\bottomrule
\end{tabular}	
\caption{List of third-party libraries used in this project}
\end{table}

Additionally there were a couple of smaller pieces of code used as part of this project that are not part of a third party library but which have been almost directly included as-is. These are documented below.

\section{Support for Multiprocessing on Python methods}
The standard Python multiprocessing library does not work with methods attached to Python objects because bound methods on an object cannot be pickled. The solution presented in ref. \cite{soMultiprocessing} was used to correct for this by adding a way to pickle and unpickle methods on Python classes.

\begin{lstlisting}[language=Python]
from copy_reg import pickle
from types import MethodType

def _pickle_method(method):
    func_name = method.im_func.__name__
    obj = method.im_self
    cls = method.im_class
    return _unpickle_method, (func_name, obj, cls)

def _unpickle_method(func_name, obj, cls):
    for cls in cls.mro():
        try:
            func = cls.__dict__[func_name]
        except KeyError:
            pass
        else:
            break
    return func.__get__(obj, cls)
    
pickle(MethodType, _pickle_method, _unpickle_method)	

\end{lstlisting}

\clearpage
\section{Gaussian Kernel}
The code used to create the Gaussian kernel used as part of the deformable convolution is taken from a Github gist by Andrew Giessel \cite{gistGaussianKernel}.

\begin{lstlisting}[language=Python]
import numpy as np
 
def makeGaussian(size, fwhm = 3, center=None):
    """ Make a square gaussian kernel.

    size is the length of a side of the square
    fwhm is full-width-half-maximum, which
    can be thought of as an effective radius.
    """
 
    x = np.arange(0, size, 1, float)
    y = x[:,np.newaxis]
    
    if center is None:
        x0 = y0 = size // 2
    else:
        x0 = center[0]
        y0 = center[1]
    
    return np.exp(-4*np.log(2) * ((x-x0)**2 + (y-y0)**2) / fwhm**2)

\end{lstlisting}
