\chapter{Background \& Objectives}

\section{Mammography}
Breast cancer is the leading cause of dead among women world wide and is the most common form of cancer found in women \cite{siegel2014cancer}. Mammography is the analysis of female breast tissue through the use of X-ray radiology with the goal of producing high resolution images of the structure within the breast. The composition of the internal structure can then be used to permit earlier detection of breast cancer.

Qualitatively speaking the composition of breast tissue can be split into four distinct categories. These are Nodular densities (corresponding to Terminal Ductal Lobular Units (TDLUs), linear densities (corresponding to ducts, vessels, and fibrous strands), homogeneous, structureless densities (corresponding to fibrous supporting tissue), and radiolucent areas (corresponding to adipose tissue).

\subsection{Risk Assessment}
Mammograms provide a non-invasive means to assess the risk of a patient developing cancer. The general goal of a mammographic risk assessment is to evaluate the risk of patient having cancer. The composition of breast tissue can be categorised using the Breast Imaging Reporting And Data System (BI-RADS)\cite{american1998breast}. BI-RADS classifies mammograms based on the density of tissue (and therefore risk) in the mammogram.

\section{Features}
\subsection{Shape Features}
\subsection{Texture Features}

\section{Dimensionality Reduction}
\subsection{Linear}
\subsection{Non Linear}

\section{Visualisation}

\section{Analysis}

\section{Research Method}
