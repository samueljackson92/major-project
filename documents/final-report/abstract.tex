\thispagestyle{empty}

\begin{center}
    {\LARGE\bf Abstract}
\end{center}

This project aimed to investigate the similarities and differences between the higher dimensional feature spaces generated from both real and synthetic mammographic images and visualise these spaces in two and three dimensions. The goals of the project were to examine where the differences between these two spaces lie and where they are similar. This was achieved through extracting a variety of different features (morphological, intensity based and textural) from both a real and synthetic mammograms and producing lower dimensional visualisations of the resulting space. Lower dimensional projections were produced by three different techniques (t-SNE, LLE and Isomap). The results of this project conclude that the synthetic mammograms under projection are dissimilar from real mammograms across all features extracted. Some similarity was shown in terms of morphological features but the similarity was not sufficient enough to mimic real mammograms in terms of their estimated risk. In terms of texture and intensity the feature space of synthetics mammograms was shown to be completely different. 
