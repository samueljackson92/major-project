\thispagestyle{empty}

\begin{center}
    {\LARGE\bf Abstract}
\end{center}

This project aimed to investigate the similarities and differences between the higher dimensional feature spaces generated from both real and synthetic mammographic images and to visualise these spaces in two or three dimensions. This was achieved by extracting a variety of different features (morphological, intensity based and textural) from real and synthetic mammograms and producing lower dimensional visualisations of the resulting space. Lower dimensional projections were created using three different techniques (t-SNE, LLE and Isomap). The results of this project conclude that the synthetic mammograms under projection are largely dissimilar from real mammograms across all features extracted. Some similarity was shown in terms of morphological features but the similarity was not sufficient to mimic real mammograms in terms of their estimated risk. In terms of the texture and intensity feature spaces synthetics mammograms were shown to be in completely different spaces to real mammograms. 
