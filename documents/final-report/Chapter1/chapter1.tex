\chapter{Background \& Objectives}

\section{Mammography}

Breast cancer is the leading cause of death among women and is the most common form of cancer found in women \cite{siegel2014cancer}. Early screening of breast cancer using mammography has been shown to reduce the mortality rate of women \cite{independent2012benefits, smith2014cancer}.

Mammography is the analysis of female breast tissue through the use of X-ray radiology with the goal of producing high resolution images of the structure within the female breast. The com- position of the parenchymal patterns and tissue density revealed by in a mammographic evaluation can be used in the early detection of breast cancer.

Qualitatively speaking the composition of breast tissue can be split into four distinct categories. These are Nodular densities (corresponding to Terminal Ductal Lobular Units (TDLUs), linear densities (corresponding to ducts, vessels, and fibrous strands), homogeneous, structureless densi- ties (corresponding to fibrous supporting tissue), and radiolucent areas (corresponding to adipose tissue). Typical markers used in the detection of cancer can are the presence of clusters of micro- calcifications, masses, architectural distortions, breast density and parenchymal patterns \cite{mccormack2006breast, sampat2005computer}.

\subsection{Risk Assessment}
Mammograms provide a non-invasive means to assess the risk of a patient developing cancer given a set of mammographic images. There are multiple classification systems used in the classification of mammographic risk.
The composition of breast tissue can be categorised using the Breast Imaging Reporting And Data System (BI-RADS) \cite{american1998breast}. BI-RADS classifies mammograms based on the density of tissue (and therefore risk) in the mammogram.

\section{Features}
\subsection{Shape Features}
\subsection{Texture Features}

\section{Dimensionality Reduction}
\subsection{Linear}
\subsection{Non Linear}

\section{Visualisation}

\section{Analysis}

\section{Research Method}
